\documentclass{article}

\usepackage{fancyhdr}
\usepackage{extramarks}
\usepackage{amsmath}
\usepackage{amsthm}
\usepackage{amsfonts}
\usepackage{tikz}
\usepackage[plain]{algorithm}
\usepackage{algpseudocode}

\usetikzlibrary{automata,positioning}

%
% Basic Document Settings
%

\topmargin=-0.45in
\evensidemargin=0in
\oddsidemargin=0in
\textwidth=6.5in
\textheight=9.0in
\headsep=0.25in

\linespread{1.1}

\pagestyle{fancy}
\lhead{\hmwkAuthorName}
\chead{\hmwkClass\ (\hmwkClassInstructor\ \hmwkClassTime): \hmwkTitle}
\rhead{\firstxmark}
\lfoot{\lastxmark}
\cfoot{\thepage}

\renewcommand\headrulewidth{0.4pt}
\renewcommand\footrulewidth{0.4pt}

\setlength\parindent{0pt}

%
% Create Problem Sections
%

\newcommand{\enterProblemHeader}[1]{
    \nobreak\extramarks{}{Problem \arabic{#1} continued on next page\ldots}\nobreak{}
    \nobreak\extramarks{Problem \arabic{#1} (continued)}{Problem \arabic{#1} continued on next page\ldots}\nobreak{}
}

\newcommand{\exitProblemHeader}[1]{
    \nobreak\extramarks{Problem \arabic{#1} (continued)}{Problem \arabic{#1} continued on next page\ldots}\nobreak{}
    \stepcounter{#1}
    \nobreak\extramarks{Problem \arabic{#1}}{}\nobreak{}
}

\setcounter{secnumdepth}{0}
\newcounter{partCounter}
\newcounter{homeworkProblemCounter}
\setcounter{homeworkProblemCounter}{1}
\nobreak\extramarks{Problem \arabic{homeworkProblemCounter}}{}\nobreak{}

%
% Homework Problem Environment
%
% This environment takes an optional argument. When given, it will adjust the
% problem counter. This is useful for when the problems given for your
% assignment aren't sequential. See the last 3 problems of this template for an
% example.
%
\newenvironment{homeworkProblem}[1][-1]{
    \ifnum#1>0
        \setcounter{homeworkProblemCounter}{#1}
    \fi
    \section{Problem \arabic{homeworkProblemCounter}}
    \setcounter{partCounter}{1}
    \enterProblemHeader{homeworkProblemCounter}
}{
    \exitProblemHeader{homeworkProblemCounter}
}

%
% Homework Details
%   - Title
%   - Due date
%   - Class
%   - Section/Time
%   - Instructor
%   - Author
%

\newcommand{\hmwkTitle}{Notes
}
%\newcommand{\hmwkDueDate}{whenever}
\newcommand{\hmwkClass}{Honors Analysis}
\newcommand{\hmwkClassTime}{MTH 327H}
\newcommand{\hmwkClassInstructor}{Professor Chekohv}
\newcommand{\hmwkAuthorName}{\textbf{Andrew Koren}}

%
% Title Page
%

\title{
    \vspace{2in}
    \textmd{\textbf{\hmwkClass:\ \hmwkTitle}}\\
    %\normalsize\vspace{0.1in}\small{Due\ on\ \hmwkDueDate\ at 3:10pm}\\
    \vspace{0.1in}\large{\textit{\hmwkClassInstructor\ \hmwkClassTime}}
    \vspace{3in}
}

\author{\hmwkAuthorName}
\date{}

\renewcommand{\part}[1]{\textbf{\large Part \Alph{partCounter}}\stepcounter{partCounter}\\}

%
% Various Helper Commands
%

% Useful for algorithms
\newcommand{\alg}[1]{\textsc{\bfseries \footnotesize #1}}

% For derivatives
\newcommand{\deriv}[1]{\frac{\mathrm{d}}{\mathrm{d}x} (#1)}

% For partial derivatives
\newcommand{\pderiv}[2]{\frac{\partial}{\partial #1} (#2)}

% Integral dx
\newcommand{\dx}{\mathrm{d}x}

% Alias for the Solution section header
\newcommand{\solution}{\textbf{\large Solution}}

% Probability commands: Expectation, Variance, Covariance, Bias
\newcommand{\E}{\mathrm{E}}
\newcommand{\Var}{\mathrm{Var}}
\newcommand{\Cov}{\mathrm{Cov}}
\newcommand{\Bias}{\mathrm{Bias}}

\begin{document}

\maketitle

\pagebreak

\begin{homeworkProblem}

Note: Linear maps require additivity $T(u+v)= Tu + Tv: u,v \in V$ and homogeneity $T(\lambda v) = \lambda(Tv): \lambda \in F, v \in V$

(i) Define $T: \mathcal{P}_4 \to \mathcal{P}_4 $ by $$(Tp)(x)=x^2p''(x).$$

Let $p,q \in \mathcal{P}_4$. Show that
$ T(p+q)(x) = (Tp)(x)+(Tq)(x)$
and
$ T(\lambda p)(x) = x^2(\lambda p)(x) = \lambda (Tp)(x) $

Pf. Show $T$ is a linear map.

$$ T(p+q)(x) =  x^2(p + q)''(x)$$ by definition of $T$
$$ x^2(p + q)''(x) = x^2(p'' + q'')(x) = x^2(p'')(x) + x^2(q'')(x) $$ by definition of differentiation, distribution in $\mathcal{P}_4$. 
$$ x^2(p'')(x) + x^2(q'')(x) = (Tp)(x)+(Tq)(x) $$ by definition of $T$. 

So $T(p+q)(x) = (Tp)(x)+(Tq)(x) $ and thus $T$ has additivity

$$T(\lambda p)(x) =  x^2(\lambda p)''(x) $$ by definition of $T$

$$ x^2(\lambda p)''(x) = \lambda x^2 p'' (x) = \lambda (Tp)(x)$$ by definition of $T$, differentiation 

So $T(\lambda p)(x) = \lambda (Tp)(x)$ and thus $T$ has homogeneity.

Since $T$ has additivity and homogeneity, $T$ is linear $\square$

(ii) Define $S:\mathcal{P}_4 \to \mathcal{P}_4$ by $$(Sp)(x) = p''(x)+x^2$$

Pf. Show that $S$ is not a linear map by example. Let $p(x)= x $ and $ q(x)=x$
$$(Sp)(x)=x''+x^2=0+x^2=x^2$$ By definition of $S$, differentiation, additive identity in $\mathcal{P}_4$
$$(Sq)(x)=x''+x^2=0+x^2=x^2$$ By definition of $S$, differentiation, additive identity in $\mathcal{P}_4$
$$(Sp)(x)+(Sq)(x)=x^2+x^2=2x^2$$
However $$(S(p+q))(x)=(x+x)''+x^2=0+x^2=x^2$$ By definition of $S$, differentiation, additive identity in $\mathcal{P}_4$, and $$x^2 \neq 2x^2$$ So $(S(p+q))(x) \neq (Sp)(x)+(Sq)(x)$, and $S$ does not have additivity, so $S$ is not linear. $\square$

(iii) $H\begin{pmatrix}
x_1 \\ x_2 \\ x_3 \\ x_4 \\
\end{pmatrix}
=
\begin{pmatrix}
x_1x_2 \\ x_1 \\ 5x_3 + x_4 \\
\end{pmatrix}
$
Pf. Show $H$ does not have additivity and is thus not linear through example.
 
Let $\vec{p} \in \mathbb{R}^4 = 
\begin{pmatrix}
1 \\ 1 \\ 1 \\ 1 \\
\end{pmatrix} $ and $\vec{q} \in \mathbb{R}^4 = 
\begin{pmatrix}
1 \\ 1 \\ 1 \\ 1 \\
\end{pmatrix} $
%Hp and Hq
$$H\vec{p}= 
H
\begin{pmatrix}
1 \\ 1 \\ 1 \\ 1 \\
\end{pmatrix}
 =
\begin{pmatrix}
1*1 \\ 1 \\ 5*1 + 1 \\
\end{pmatrix}
=
\begin{pmatrix}
1 \\ 1 \\ 6 \\
\end{pmatrix}
$$
$$H\vec{q}= H
\begin{pmatrix}
1 \\ 1 \\ 1 \\ 1 \\
\end{pmatrix}
=
\begin{pmatrix}
1*1 \\ 1 \\ 5*1 + 1 \\
\end{pmatrix}
=
\begin{pmatrix}
1 \\ 1 \\ 6 \\
\end{pmatrix}
$$
%Hp+Hq
$$H(\vec{p})+H(\vec{q})=
\begin{pmatrix}
1 \\ 1 \\ 6 \\
\end{pmatrix}
+
\begin{pmatrix}
1 \\ 1 \\ 6\\
\end{pmatrix}
=
\begin{pmatrix}
1+1 \\ 1+1 \\ 6 + 6\\
\end{pmatrix}
=
\begin{pmatrix}
2\\ 2\\ 12\\
\end{pmatrix}
$$

$$
H(\vec{p}+\vec{q})
=
H
\begin{pmatrix}
1+1\\ 1+1\\  1+1\\  1+1\\
\end{pmatrix}
=
H\begin{pmatrix}
2\\ 2\\  2\\  2\\
\end{pmatrix}
=
\begin{pmatrix}
2*2 \\ 2 \\ 5(2) + 2\\
\end{pmatrix}
=
\begin{pmatrix}
4\\ 2\\ 12 \\
\end{pmatrix}
$$
However $\begin{pmatrix}
4\\ 2\\ 12 \\
\end{pmatrix} \neq \begin{pmatrix}
2\\ 2\\ 12\\
\end{pmatrix}$, so $H\vec{p}+H\vec{q} \neq H(\vec{p}+\vec{q})$, so $H$ does not have additivity and is not linear. $\square$

(iv) Pf. Show $I$ is not linear since $I$ does not have homogeneity.

Let $\vec{o} \in \mathbb{R}^4 $ be the zero vector $ 
\begin{pmatrix}
0 \\ 0 \\ 0\\ 0\\
\end{pmatrix}
$
and $\lambda \in \mathbb{R} = 0$
$$
H\vec{o} =\lambda H \begin{pmatrix} 0 \\ 0 \\ 0\\ 0\\ \end{pmatrix} = 
\begin{pmatrix}
0-3*0 \\ 0 \\ 0+7 \\ 5*0+0
\end{pmatrix}
=
\begin{pmatrix}
0 \\ 0 \\ 7 \\ 0 \\
\end{pmatrix}
$$
Note that
$$H(\lambda \vec{o}) = H(0 \vec{o}) = H\vec{o}$$
Since any vector multiplied by zero is zero

$$
\lambda H\vec{o} =\lambda H \begin{pmatrix} 0 \\ 0 \\ 0\\ 0\\ \end{pmatrix} = 
\lambda \begin{pmatrix}
0-3*0 \\ 0 \\ 0+7 \\ 5*0+0
\end{pmatrix}
=
0*\begin{pmatrix}
0 \\ 0 \\ 7 \\ 0 \\
\end{pmatrix}
=
\begin{pmatrix}
0 \\ 0 \\ 0 \\ 0\\
\end{pmatrix}
$$

Since $\begin{pmatrix}
0 \\ 0 \\ 0 \\ 0\\
\end{pmatrix} \neq \begin{pmatrix}
0 \\ 0 \\ 7 \\ 0 \\
\end{pmatrix}$, $\lambda H\vec{o} \neq H(\lambda \vec{o}) $, H is not homogeneous and thus not linear. $\square$

(v) Pf. Show that J has both additivity and homogeneity, and is thus linear.

Let $\vec{v} \in \mathbb{R}^4 = 
\begin{pmatrix}
v_1 \\ v_2 \\ v_3 \\ v_4 \\
\end{pmatrix}$ 
and $\vec{w} = 
\begin{pmatrix}
w_1 \\ w_2 \\ w_3 \\ w_4 \\
\end{pmatrix}
$
Additivity. 
$$J(\vec{v}+\vec{w})
=J
\begin{pmatrix}
v_1+w_1 \\ v_2+w_2 \\ v_3+w_3 \\ v_4+w_4 \\
\end{pmatrix}  
=
\begin{pmatrix}
v_1+w_1 - 3(v_2+w_2) \\ v_1+w_1 \\ 5(v_3+w_3) + v_4+w_4 \\
\end{pmatrix}  
$$
$$
J\vec{v}=J
\begin{pmatrix}
v_1 \\ v_2 \\ v_3 \\ v_4 \\
\end{pmatrix} = 
\begin{pmatrix}
v_1 - 3v_2 \\ v_1 \\ 5v_3 + v_4 \\
\end{pmatrix}  
$$

$$
J\vec{w}=J
\begin{pmatrix}
w_1 \\ w_2 \\ w_3 \\ w_4 \\
\end{pmatrix} = 
\begin{pmatrix}
w_1 - 3w_2 \\ w_1 \\ 5w_3 + w_4 \\
\end{pmatrix}  
$$
$$
J\vec{w}+J\vec{v} = \begin{pmatrix}
w_1 - 3w_2 \\ w_1 \\ 5w_3 + w_4 \\
\end{pmatrix} + \begin{pmatrix}
v_1 - 3v_2 \\ v_1 \\ 5v_3 + v_4 \\
\end{pmatrix}=\begin{pmatrix}
w_1 - 3w_2 + v_1 - 3v_2 \\ w_1 + w2\\ 5w_3 + w_4 + 5v_3 + w_4 \\
\end{pmatrix} = \begin{pmatrix}
v_1+w_1 - 3(v_2+w_2) \\ v_1+w_1 \\ 5(v_3+w_3) + v_4+w_4 \\
\end{pmatrix}  
$$
So $J\vec{w}+J\vec{v} = J\vec{w} \vec{v}$ and $J$ has additivity.

Homogeneity.
$$\lambda J\vec{v}=\lambda \begin{pmatrix}
v_1 - 3v_2 \\ v_1 \\ 5v_3 + v_4 \\
\end{pmatrix} =
\begin{pmatrix}
\lambda v_1 - \lambda 3v_2 \\ \lambda v_1 \\ \lambda 5v_3 + \lambda v_4 \\
 \end{pmatrix}
$$
$$
J(\lambda \vec{v}) = J(\lambda \begin{pmatrix} v_1 \\ v_2 \\ v_3 \\ v_4 \\\end{pmatrix})=J\begin{pmatrix} \lambda v_1 \\ \lambda v_2 \\ \lambda v_3 \\ \lambda v_4 \\ \end{pmatrix}= \begin{pmatrix}
\lambda v_1 -  3\lambda v_2 \\ \lambda v_1 \\ 5 \lambda v_3 + \lambda v_4 \\
 \end{pmatrix} = \begin{pmatrix}
\lambda v_1 - \lambda 3v_2 \\ \lambda v_1 \\ \lambda 5v_3 + \lambda v_4 \\
 \end{pmatrix}
$$

\end{homeworkProblem}

\pagebreak

\end{document}